\chapter{PENGUJIAN DAN ANALISIS}
\label{chap:pengujiananalisis}

% Ubah bagian-bagian berikut dengan isi dari pengujian dan analisis

Pada bab ini akan dipaparkan skenario pengujian yang  dilakukan berdasarkan dari metodologi yang telah dibahas sebelumnya beserta dengan pembahasannya.

\section{Skenario Pengujian}
\label{sec:skenariopengujian}

Pengujian ini akan dilakukan dengan menggunakan Intel NUC 11 Performance Kit yang telah terhubung ke internet untuk mengakses program melalui browser. Dibutuhkan juga tambahan koneksi dengan webcam eksternal sebagai media pengambilan data yang akan diproses pada program, serta keyboard, mouse, dan monitor untuk memudahkan dalam mengoperasikan program. Adapun pengujian ini akan mengikuti beberapa skenario - skenario pengujian yang ditentukan sebagai berikut:

\begin{enumerate}
  \item Skenario pengujian berdasarkan bentuk model
  \item Skenario pengujian berdasarkan kondisi pencahayaan yang berbeda, yaitu 35 lux, 80 lux, 125 lux. 
  \item Skenario pengujian berdasarkan jarak subjek yang berbeda, yaitu 180 cm, 240 cm, 300 cm.
  \item Skenario pengujian dengan menggunakan subjek yang berbeda selain penulis
  \item Skenario pengujian pembentukan kalimat dan koversi menjadi media suara
\end{enumerate}

\section{Pengujian Bentuk Model}
\label{sec:analisismodel}

Pengujian bentuk model penerjemah bahasa isyarat Indonesia (BISINDO) dilakukan dengan melakukan perubahan struktur dari \emph{layer} yang digunakan, baik dari segi pemilihan tipe \emph{layer}, fungsi \emph{activation}, serta jumlah unit aktivasi yang digunakan. Pengujian ini didasari dari struktur model yang telah dibahas pada sub bab \ref{sec:metodologipose}. Serangkaian model yang diuji akan dilihat bagaimana performa hasil dari \emph{training} model, melalui grafik \emph{accuracy} dan \emph{loss} yang dihasilkan. Digunakan \emph{confusion matrix} untuk mengamati bagaimana hasil dari pengujian model terhadap serangkaian data gerakan bahasa isyarat dengan membandingkan hasil prediksi model dengan data aktual yang ada untuk masing - masing kosakata atau \emph{class} yang ada. Berdasarkan \emph{confusion matrix} ini juga dapat dihasilkan matrix evaluasi berupa \emph{accuracy}, \emph{precision}, \emph{recall}, dan \emph{F1-score}. Seiap model yang diuji menggunakan partisi data \emph{training} dan validasi yang sama, yaitu dengan perbandingan 70:30. Hal ini menunjukkan bahwa dari keseluruhan data yang digunakan, akan terdapat 70\% data \emph{training} dan 30\% data validasi. Untuk setiap kosakata yang digunakan sesuai dengan yang telah dijelaskan pada sub bab\ref{sec:metodologidataset}. Keseluruhan model akan dilatih sebanyak 12 \emph{epoch}. Pengujian ini bertujuan untuk melihat bagaimana perubahan struktur dari \emph{layer} akan berpengaruh pada performa model penerjemah yang dihasilkan.

\subsection{Model Pertama}
\label{sec:analisismodel1}
Model pertama menggunakan 2 buah \emph{layer} LSTM. \emph{Layer} LSTM pertama menggunakan fungsi aktivasi \emph{relu} dan dengan unit aktivasi bernilai 128 dan \emph{layer} LSTM kedua menggunakan fungsi aktivasi \emph{relu} dan dengan unit aktivasi bernilai 64. Untuk setiap \emph{layer} LSTM akan diikuti dengan \emph{layer} Dropout bernilai 0.5 untuk mencegah nilai \emph{weight} yang terlalu tinggi. Setelah serangkaian \emph{layer} LSTM, diikuti dengan \emph{layer} Dense dengan fungsi aktivasi \emph{relu} dan dengan unit aktivasi bernilai 32. Struktur lengkap dari model ini dapat dilihat pada \ref{fig:model1-struktur}.

\begin{figure}[H]
  \centering

  \includegraphics[scale=0.5]{gambar/bab4-uji-model-worst-model.png}

  \caption{Struktur model pertama}
  \label{fig:model1-struktur}
\end{figure}

Berdasarkan dari \emph{training} yang telah dilakukan didapatkan bahwa model menghasilkan akurasi \emph{training} bernilai 0.89 dan akurasi validasi bernilai 0.71. Data ini menunjukkan bahwa model memiliki akurasi yang cukup baik. Untuk nilai \emph{loss training} bernilai cukup tinggi, yaitu bernilai 2.81 dan \emph{loss} validasi bernilai 0.76 yang cukup rendanh jika dibandingkan dengan nilai \emph{loss training}. Nilai dari \emph{loss training} terlihat melonjak di \emph{epoch} terakhir, yaitu pada \emph{epoch} 12. Hal ini selaras dengan penurunan \emph{accuracy} model setelah \emph{epoch} 10. Grafik akurasi dan \emph{loss} dapat dilihat pada gambar \ref{fig:model1-train-acc} dan gambar \ref{fig:model1-train-loss}.

\begin{figure}[H]
  \centering

  \includegraphics[scale=0.75]{gambar/bab4-uji-model-worst-acc.png}

  \caption{Hasil \emph{accuracy} model pertama}
  \label{fig:model1-train-acc}
\end{figure}

\begin{figure}[H]
  \centering

  \includegraphics[scale=0.75]{gambar/bab4-uji-model-worst-loss.png}

  \caption{Hasil \emph{loss} model pertama}
  \label{fig:model1-train-loss}
\end{figure}

Kemudian berdasarkan model yang telah dihasilkan, dilakukan pengujian dengan dataset \emph{testing} yang menghasilkan \emph{confusion matrix}. Dapat dilihat pada gambar \ref{fig:model1-cf}, untuk kosakata "maaf", "tolong", "rumah", "delete", dan "translate" menghasilkan prediksi yang tepat untuk keseluruhan dataset \emph{testing} yang diujikan. Namun, kosakata "nama", "saya", "siapa", dan "standby" menghasilkan beberapa prediksi yang tidak sesuai dengan dataset \emph{testing}. Kosakata "nama" menghasilkan 6 prediksi tepat dan 4 prediksi yang kurang tepat (1 dataset \emph{testing} bernilai "siapa" dan 3 dataset \emph{testing} bernilai "rumah"). Kosakata "saya" menghasilkan 10 prediksi tepat dan 2 prediksi yang kurang tepat (dataset \emph{testing} bernilai "translate"). Kosakata "siapa" menghasilkan 3 prediksi tepat dan 5 prediksi yang kurang tepat (2 dataset \emph{testing} bernilai "tolong",  2 dataset \emph{testing} bernilai "saya", dan 1 dataset \emph{testing} bernilai "translate"). Kosakata "standby" menghasilkan 7 prediksi tepat dan 1 prediksi yang kurang tepat (dataset \emph{testing} bernilai "saya"). Adapun berdasarkan hasil \emph{confusion matrix} ini didapat matrix evaluasi berupa \emph{accuracy}, \emph{precision}, \emph{recall}, dan \emph{F1-score} yang dapat dilihat pada tabel \ref{tb:model1stat}. Rata - rata nilai \emph{accuracy} sebesar 0.85, rata - rata nilai \emph{recall} sebesar 0.85, dan rata - rata nilai \emph{F1-score} sebesar 0.84.

\begin{figure}[H]
  \centering

  \includegraphics[scale=0.6]{gambar/bab4-uji-model-worst-cf.png}

  \caption{\emph{Confusion} model pertama}
  \label{fig:model1-cf}
\end{figure}

\begin{longtable}{|c|c|c|c|}
  \caption{Matrix Evaluasi Model 1}
  \label{tb:model1stat}                                   \\
  \hline
  \rowcolor[HTML]{C0C0C0}
  \textbf{Kosakata} & \textbf{\emph{Accuracy}} & \textbf{\emph{Precision}} & \textbf{\emph{F1-Score}} \\
  \hline
  Maaf              & 1.00                        & 1.00                   & 1.00                \\
  Tolong            & 0.80                        & 1.00                   & 0.89                \\
  Nama              & 1.00                        & 0.60                   & 0.75                \\
  Saya              & 0.77                        & 0.83                   & 0.80                \\
  Siapa             & 0.75                        & 0.38                   & 0.50                \\
  Rumah             & 0.62                        & 1.00                   & 0.77                \\
  Delete            & 1.00                        & 1.00                   & 1.00                \\
  Standby           & 1.00                        & 0.88                   & 0.93                \\
  Translate         & 0.70                        & 1.00                   & 0.82                \\
  \hline
\end{longtable}

\subsection{Model Kedua}
\label{sec:analisismodel2}

Model kedua diawali dengan \emph{layer} \textit{TimeDistributed} yang di dalamnya terdapat \emph{layer} \textit{Dense} dengan fungsi aktivasi '\textit{tanh}' dan unit aktivasi bernilai 128. Kemudian dilanjutkan dengan 1 buah \emph{layer} LSTM yang menggunakan fungsi aktivasi \emph{tanh} dan dengan unit aktivasi bernilai 64. \emph{Layer} LSTM akan diikuti dengan \emph{layer} Dropout bernilai 0.5 untuk mencegah nilai \emph{weight} yang terlalu tinggi. Setelah serangkaian \emph{layer} LSTM, diikuti dengan \emph{layer} Dense dengan fungsi aktivasi \emph{relu} dan dengan unit aktivasi bernilai 32. Struktur lengkap dari model ini dapat dilihat pada \ref{fig:model2-struktur}.

\begin{figure}[H]
  \centering

  \includegraphics[scale=0.6]{gambar/bab4-uji-model-second-model.png}

  \caption{Struktur model kedua}
  \label{fig:model2-struktur}
\end{figure}

Berdasarkan dari \emph{training} yang telah dilakukan didapatkan bahwa model menghasilkan akurasi \emph{training} bernilai 0.96 dan akurasi validasi bernilai 0.93. Data ini menunjukkan bahwa model memiliki akurasi yang baik. Terdapat penurunan pada akurasi validasi pada \emph{epoch} terakhir, tetapi kebalikannya untuk akurasi \emph{training} mengalami kenaikan melebihi akurasi validasi. Untuk nilai \emph{loss training} bernilai 0.32 dan \emph{loss} validasi bernilai 0.25 yang cukup rendah jika dibandingkan dengan nilai \emph{loss training}. Data ini menunjukkan bahwa model memiliki \emph{error} prediksi yang kecil. Grafik akurasi dan \emph{loss} dapat dilihat pada gambar \ref{fig:model2-train-acc} dan gambar \ref{fig:model2-train-loss}.

\begin{figure}[H]
  \centering

  \includegraphics[scale=0.75]{gambar/bab4-uji-model-second-acc.png}

  \caption{Hasil \emph{accuracy} model kedua}
  \label{fig:model2-train-acc}
\end{figure}

\begin{figure}[H]
  \centering

  \includegraphics[scale=0.75]{gambar/bab4-uji-model-second-loss.png}

  \caption{Hasil \emph{loss} model kedua}
  \label{fig:model2-train-loss}
\end{figure}


Kemudian berdasarkan model yang telah dihasilkan, dilakukan pengujian dengan dataset \emph{testing} yang menghasilkan \emph{confusion matrix}. Dapat dilihat pada gambar \ref{fig:model2-cf}, untuk kosakata "tolong", "nama", "rumah", "delete", dan "translate" menghasilkan prediksi yang tepat untuk keseluruhan dataset \emph{testing} yang diujikan. Namun, kosakata "saya", "siapa", dan "standby" menghasilkan beberapa prediksi yang tidak sesuai dengan dataset \emph{testing}. Kosakata "maaf" menghasilkan 10 prediksi tepat dan 1 prediksi yang kurang tepat (dataset \emph{testing} bernilai "standb\\y"). Kosakata "saya" menghasilkan 11 prediksi tepat dan 2 prediksi yang kurang tepat (dataset \emph{testing} bernilai "siapa"). Kosakata "siapa" menghasilkan 5 prediksi tepat dan 1 prediksi yang kurang tepat (dataset \emph{testing} bernilai "saya"). Kosakata "standby" menghasilkan 7 prediksi tepat dan 1 prediksi yang kurang tepat (dataset \emph{testing} bernilai "siapa"). Dapat dilihat bahwa terdapat peningkatan performa dari model pertama dengan model kedua. Kosakata yang berhasil diprediksi sesuai dengan dataset \emph{testing} yang digunakan lebih banyak dibandingkan dengan model sebelumnya, yaitu 5 kosakata. Peningkatan ini disebabkan karena perubahan struktur dari model dengan adanya penggunaan \emph{layer Time Distributed} yang di dalamnya diisi dengan \emph{layer} Dense di lapisan awal model. Pengurangan pengunaan 1 \emph{layer} LSTM dapat dengan menggunakan fungsi aktivasi \emph{tanh}. Adapun berdasarkan \emph{confusion matrix} ini didapat matrix evaluasi berupa \emph{accuracy}, \emph{precision}, \emph{recall}, dan \emph{F1-score} yang dapat dilihat pada tabel \ref{tb:model2stat}. Rata - rata nilai \emph{accuracy} sebesar 0.94, rata - rata nilai \emph{recall} sebesar 0.94, dan rata - rata nilai \emph{F1-score} sebesar 0.94. Dapat dilihat berdasarkan matrix evaluasi ini, terdapat peningkatan performa pada model kedua jika dibandingkan dengan model pertama. 

\begin{figure}[H]
  \centering

  \includegraphics[scale=0.6]{gambar/bab4-uji-model-second-cf.png}

  \caption{\emph{Confusion} model kedua}
  \label{fig:model2-cf}
\end{figure}

\begin{longtable}{|c|c|c|c|}
  \caption{Matrix Evaluasi Model 2}
  \label{tb:model2stat}                                   \\
  \hline
  \rowcolor[HTML]{C0C0C0}
  \textbf{Kosakata} & \textbf{\emph{Accuracy}} & \textbf{\emph{Precision}} & \textbf{\emph{F1-Score}} \\
  \hline
  Maaf              & 0.91                        & 1.00                   & 0.95                \\
  Tolong            & 1.00                        & 1.00                   & 1.00                \\
  Nama              & 1.00                        & 1.00                   & 1.00                \\
  Saya              & 0.85                        & 0.92                   & 0.88                \\
  Siapa             & 0.93                        & 0.62                   & 0.71                \\
  Rumah             & 1.00                        & 1.00                   & 1.00                \\
  Delete            & 1.00                        & 1.00                   & 1.00                \\
  Standby           & 0.88                        & 0.88                   & 0.88                \\
  Translate         & 1.00                        & 1.00                   & 1.00                \\
  \hline
\end{longtable}

\subsection{Model Ketiga}
\label{sec:analisismodel3}

Model ketiga merupakan gabungan dari struktur antara model 1 dan model 2. Model ini diawali dengan layer pertama berupa \emph{layer} \textit{TimeDistributed} yang di dalamnya terdapat \emph{layer} \textit{Dense} dengan fungsi aktivasi '\textit{tanh}' dan unit aktivasi bernilai 128. Selanjutnya diikuti dengan 2 buah \emph{layer} LSTM. \emph{Layer} LSTM pertama menggunakan fungsi aktivasi \emph{tanh} dengan unit aktivasi bernilai 128 dan \emph{Layer} LSTM kedua menggunakan fungsi aktivasi \emph{tanh} dengan unit aktivasi bernilai 64. Kedua \emph{layer} LSTM ini diikuti dengan \emph{layer} Dropout bernilai 0.5 untuk mencegah nilai \emph{weight} yang terlalu tinggi. Dilanjutkan dengan \emph{layer} Dense dengan fungsi aktivasi \emph{relu} dan unit aktivasi bernilai 32. Struktur lengkap dari model ini dapat dilihat pada gambar \ref{fig:model3-struktur}

\begin{figure}[H]
  \centering

  \includegraphics[scale=0.5]{gambar/bab4-uji-model-best-model.png}

  \caption{Struktur model ketiga}
  \label{fig:model3-struktur}
\end{figure}

Berdasarkan dari \emph{training} yang telah dilakukan didapatkan bahwa model menghasilkan akurasi \emph{training} bernilai 0.94 dan akurasi validasi bernilai 1.00. Data ini menunjukkan bahwa model memiliki akurasi yang sangat baik. Untuk nilai \emph{loss training} bernilai 0.3 dan \emph{loss} validasi bernilai 0.16 yang rendah jika dibandingkan dengan nilai \emph{loss training}. Data ini menunjukkan bahwa model memiliki \emph{error} prediksi yang kecil. Hasil dari \emph{training} model ini menunjukkan bahwa model telah dapat mempelajari dataset dengan baik. Ketika dilakukan pengujian dengan menggunakan data validasi, dapat dilihat bahwa model memiliki akurasi yang lebih tinggi dan tingkat \emph{loss} yang lebih rendah dibandingkan dengan pengujian yang menggunakan data \emph{training} itu sendiri. Grafik akurasi dan \emph{loss} dapat dilihat pada gambar \ref{fig:model3-train-acc} dan gambar \ref{fig:model3-train-loss}.

\begin{figure}[H]
  \centering

  \includegraphics[scale=0.75]{gambar/bab4-uji-model-best-acc.png}

  \caption{Hasil \emph{accuracy} model ketiga}
  \label{fig:model3-train-acc}
\end{figure}

\begin{figure}[H]
  \centering

  \includegraphics[scale=0.75]{gambar/bab4-uji-model-best-loss.png}

  \caption{Hasil \emph{loss} model ketiga}
  \label{fig:model3-train-loss}
\end{figure}


Kemudian berdasarkan model yang telah dihasilkan, dilakukan pengujian dengan dataset \emph{testing} yang menghasilkan \emph{confusion matrix}. Dapat dilihat pada gambar \ref{fig:model3-cf}, untuk kosakata "maaf", "tolong", "nama", "saya",  menghasilkan prediksi yang tepat untuk keseluruhan dataset \emph{testing} yang diujikan. Namun, pada kosakata "siapa" menghasilkan beberapa prediksi yang tidak sesuai dengan dataset \emph{testing}. Kosakata "siapa" menghasilkan 7 prediksi tepat dan 1 prediksi yang kurang tepat (dataset \emph{testing} bernilai "saya"). Dapat dilihat bahwa terdapat peningkatan performa dari model ketiga dibandingkan dengan model pertama dan kedua. Peningkatan ini disebabkan karena perubahan struktur dari model dengan penggunaan \emph{layer Time Distributed} yang didalamnya diisi dengan \emph{layer} Dense di lapisan awal model
. Kemudian dilanjutkan dengan 2 buah \emph{layer} LSTM yang menggunakan fungsi aktivasi \emph{tanh} Adapun berdasarkan \emph{confusion matrix} ini didapat matrix evaluasi berupa \emph{accuracy}, \emph{precision}, \emph{recall}, dan \emph{F1-score} yang dapat dilihat pada tabel \ref{tb:model3stat}. Rata - rata nilai \emph{accuracy} sebesar 0.99, rata - rata nilai \emph{recall} sebesar 0.99, dan rata - rata nilai \emph{F1-score} sebesar 0.99. Dapat dilihat berdasarkan matrix evaluasi ini, terdapat peningkatan performa pada model ketiga jika dibandingkan dengan model pertama dan kedua.

\begin{figure}[H]
  \centering

  \includegraphics[scale=0.6]{gambar/bab4-uji-model-best-cf.png}

  \caption{\emph{Confusion} model pertama}
  \label{fig:model3-cf}
\end{figure}

\begin{longtable}{|c|c|c|c|}
  \caption{Matrix Evaluasi Model 3}
  \label{tb:model3stat}                                   \\
  \hline
  \rowcolor[HTML]{C0C0C0}
  \textbf{Kosakata} & \textbf{\emph{Accuracy}} & \textbf{\emph{Precision}} & \textbf{\emph{F1-Score}} \\
  \hline
  Maaf              & 1.00                        & 1.00                   & 1.00                \\
  Tolong            & 1.00                        & 1.00                   & 1.00                \\
  Nama              & 1.00                        & 1.00                   & 1.00                \\
  Saya              & 0.92                        & 1.00                   & 0.96                \\
  Siapa             & 1.00                        & 0.82                   & 0.93                \\
  Rumah             & 1.00                        & 1.00                   & 1.00                \\
  Delete            & 1.00                        & 1.00                   & 1.00                \\
  Standby           & 1.00                        & 1.00                   & 1.00                \\
  Translate         & 1.00                        & 1.00                   & 1.00                \\
  \hline
\end{longtable}

\section{Pengujian Kondisi Cahaya}
\label{sec:analisiscahaya}

Pada pengujian kondisi cahaya ini dilakukan untuk memahami bagaimana performa model dalam kondisi intensitas cahaya  yang berbeda - beda. Adapun intensitas cahaya yang akan digunakan pada pengujian ini adalah 35 lux (kondisi ruangan gelap), 80 lux (kondisi ruangan remang - remang), dan 125 lux (kondisi ruangan terang). Variasi intensitas cahaya ini dipilih karena merupakan intensitas cahaya yang umum ditemukan pada ruangan tertutup. Kondisi ruangan dengan intensitas cahaya masing - masing dapat dilihat pada tabel \ref{tb:kondisicahaya}. Pengambilan nilai intensitas cahaya ini dilakukan dengan menggunakan \emph{lux meter} yang telah dikalibrasi untuk memastikan ketelitian hasil pengukuran.

Model penerjemah bahasa Indonesia (BISINDO) yang akan digunakan pada pengujian ini adalah model pada bagian \ref{sec:analisismodel3} karena merupakan model yang menghasilkan klasifikasi yang terbaik jika dibandingkan dengan model lainnya. Untuk setiap intensitas cahaya akan dilakukan pengujian sebanyak tiga kali dengan jarak terhadap kamera sebesar 300 cm. Pada setiap pengujian akan dicari hasil klasifikasi model, waktu yang dibutuhkan model untuk menghasilkan klasifikasi bahasa isyarat berdasarkan data koordinat yang diberikan(\emph{processing time}), dan waktu total yang dibutuhkan dalam menghasilkan klasifikasi bahasa isyarat (\emph{complete time}).  

\begin{longtable}{|c|c|}
  \caption{Variasi Kondisi Cahaya}
  \label{tb:kondisicahaya}                                   \\
  \hline
  \rowcolor[HTML]{C0C0C0}
  \textbf{Intensitas Cahaya} & \textbf{Gambar Kondisi}  \\
  \hline
  35 lux            &  \includegraphics[scale=0.3]{gambar/bab4-gelap.png}                \\
  \hline
  80 lux            & \includegraphics[scale=0.3]{gambar/bab4-remang.png}                 \\
  \hline
  125 lux            & \includegraphics[scale=0.3]{gambar/bab4-terang.png}                 \\
  \hline
\end{longtable}

\subsection{Pengujian Model di Kondisi Cahaya 35 lux}
\label{sec:analisiscahaya1}

\begin{longtable}{|c|c|c|c|}
  \caption{Pengujian Pertama Model di Kondisi Cahaya 35 lux}
  \label{tb:prediksigelap1}                                   \\
  \hline
  \rowcolor[HTML]{C0C0C0}
  \textbf{Kosakata} & \textbf{Klasifikasi Model} & \textbf{\emph{Processing Time}} & \textbf{\emph{Complete Time}}\\
  \hline
  Maaf              & Maaf                        & 0.09208822250366211                           & 2.8735828399658203                                  \\
  Tolong            & Tolong                        & 0.09208822250366211                           & 2.9096102714538574                                  \\
  Nama              & Nama                        & 0.08932256698608398                           & 3.3274698257446294                                  \\
  Saya              & Saya                        & 0.09866833686828613                           & 2.8822731971740723                                  \\
  Siapa              & Siapa                        & 0.08973073959350586                           & 2.8529834747314453                                  \\
  Rumah             & \textcolor{red}{Delete}                        & 0.09612202644348145                           & 2.8643417358398438                                  \\
  Delete            & Delete                        & 0.09255480766296387                           & 2.824845314025879                                  \\
  Standby           & Standby                        & 0.09247112274169922                           & 2.8762435913085938                                  \\
  Translate         & Translate                        & 0.09525346755981445                           & 2.9262471199035645                                  \\
  \hline
\end{longtable}

\begin{longtable}{|c|c|c|c|}
  \caption{Pengujian Kedua Model di Kondisi Cahaya 35 lux}
  \label{tb:prediksigelap2}                                   \\
  \hline
  \rowcolor[HTML]{C0C0C0}
  \textbf{Kosakata} & \textbf{Klasifikasi Model} & \textbf{\emph{Processing Time}} & \textbf{\emph{Complete Time}} \\
  \hline
  Maaf              & Maaf                        & 0.10043215751647949                           & 3.328549861907959                                  \\
  Tolong            & Tolong                        & 0.0972287654876709                           & 3.3190798759460445                                  \\
  Nama              & \textcolor{red}{Standby}                        & 0.08973073959350586                           & 2.914302349090576                                  \\
  Saya              & Saya                        & 0.1007227897644043                           & 2.789583206176758                                  \\
  Siapa              & Siapa                        & 0.08973073959350586                           & 3.4017491340637203                                  \\
  Rumah             & Rumah                        & 0.10046267509460449                           & 2.94283390045166                                  \\
  Delete            & Delete                        & 0.09255480766296387                           & 2.9163265228271484                                  \\
  Standby           & Standby                        & 0.09181642532348633                           & 2.8268051147460938                                  \\
  Translate         & Translate                        & 0.08920478820800781                           & 3.3912491798400874                                  \\
  \hline
\end{longtable}

\begin{longtable}{|c|c|c|c|}
  \caption{Pengujian Ketiga Model di Kondisi Cahaya 35 lux}
  \label{tb:prediksigelap3}                                   \\
  \hline
  \rowcolor[HTML]{C0C0C0}
  \textbf{Kosakata} & \textbf{Klasifikasi Model} & \textbf{\emph{Processing Time}} & \textbf{\emph{Complete Time}}\\
  \hline
  Maaf              & Maaf                        & 0.09208822250366211                           & 2.9158973693847656                                  \\
  Tolong            & Tolong                        & 0.0972287654876709                           & 3.0014920234680176                                  \\
  Nama              & Nama                        & 0.09724950790405273                           & 2.8235292434692383                                  \\
  Saya              & Saya                        & 0.08916234970092773                           & 2.9606938362121578                                  \\
  Siapa              & Siapa                        & 0.08973073959350586                           & 2.8408455848693848                                  \\
  Rumah             & \textcolor{red}{Delete}                        & 0.10066267509460449                           & 3.36672306060791                                  \\
  Delete            & Delete                        & 0.09255480766296387                           & 3.4308242797851562                                  \\
  Standby           & Standby                        & 0.09866833686828613                           & 3.3450722694396973                                  \\
  Translate         & Translate                        & 0.08920478820800781                           & 2.830524444580078                                  \\
  \hline
\end{longtable}

Berdasarkan tiga pengujian yang telah dilakukan, didapatkan bahwa hampir keseluruhan klasifikasi model yang sesuai dengan \emph{class} kosakata. Namun, terdapat beberapa kesalahan model dalam melakukan klasifikasi. Dapat dilihat pada tabel \ref{tb:prediksigelap1} untuk kosakata "Rumah" diklasifikasikan sebagai "Delete". Kesalahan ini juga terjadi pada tabel \ref{tb:prediksigelap3}. Kemiripan antara gerakan untuk isyarat kosakata "Rumah" dan "Delete", serta kamera yang kesulitan menangkap sepenuhnya pose yang baik menjadi penyebab adanya kesalahan klasifikasi ini. Pada tabel \ref{tb:prediksigelap2}, untuk isyarat kosakata "Nama" diklasifikasikan sebagai "Standby". Kesalahan klasifikasi ini juga disebabkan oleh Mediapipe yang tidak sepenuhnya menangkap pose yang sesuai dengan kosakata "Nama". Pada pengujian ini,  model dapat melakukan klasifikasi yang baik meskipun pengguna berada di dalam kondisi ruangan yang terbilang gelap. Berdasakan data pada tabel \ref{tb:prediksigelap1}, tabel \ref{tb:prediksigelap2}, tabel \ref{tb:prediksigelap3} menunjukkan bahwa secara keseluruhan model memiliki akurasi klasifikasi sebesar 89.9\%. 

Apabila dilihat berdasarkan waktu pemrosesan, rata - rata waktu yang dibutuhkan model untuk menghasilkan klasifikasi bahasa isyarat (\emph{processing time}) adalah 0.093 detik dan rata - rata waktu yang dibutuhkan dalam menghasilkan klasifikasi bahasa isyarat (\emph{complete time}) adalah 3.025 detik. Pada proses pengujian yang dilakukan secara \emph{real time} ini, model memerlukan waktu yang terbilang cepat dalam memproses serangkaian data koordinat yang diberikan. Program penerjemah juga telah mampu menyelesaikan proses klasifikasi dengan cepat. Hal ini menunjukkan bahwa pada kondisi ruangan dengan intensitas cahaya yang cukup gelap, tidak berpengaruh secara signifikan terhadap \emph{processing time} dan \emph{complete time}.

Pada pengujian di intensitas cahaya 35 lux, didapat bahwa pengguna tidak dapat melakukan gerakan bahasa isyarat secara cepat. Hal ini disebabkan oleh kondisi ruangan yang gelap ini mempengaruhi kinerja kemmapuan kamera dalam menangkap tiap \emph{frame}. Namun, kemampuan \emph{framework} Mediapipe untuk mendeteksi pose pengguna dan melakukan ekstraksi koordinat berdasarkan \emph{landmark} yang ada masih dapat berjalan dengan baik pada kondisi ruangan dengan intensitas cahaya yang kurang baik. Hal ini membantu proses klasifikasi model karena data koordinat yang diproses untuk menghasilkan klasifikasi bahasa isyarat tetap dapat dieksraksi dengan tepat dan tidak memiliki banyak error atau data koordinat kosong (bernilai 0). 

\subsection{Pengujian Model di Kondisi Cahaya 80 lux}
\label{sec:analisiscahaya2}

\begin{longtable}{|c|c|c|c|}
  \caption{Pengujian Pertama Model di Kondisi Cahaya 80 lux}
  \label{tb:prediksiremang1}                                   \\
  \hline
  \rowcolor[HTML]{C0C0C0}
  \textbf{Kosakata} & \textbf{Klasifikasi Model} & \textbf{\emph{Processing Time}} & \textbf{\emph{Complete Time}}\\
  \hline
  Maaf              & Maaf                        & 0.09171247482299805                           & 2.749907970428467                                  \\
  Tolong            & Tolong                        & 0.09063005447387695                           & 3.3760499954223637                                  \\
  Nama              & Nama                        & 0.09208822250366211                           & 2.930173873901367                                  \\
  Saya              & Saya                        & 0.09037542343139648                           & 2.723050117492676                                  \\
  Siapa              & Siapa                        & 0.09206461906433105                           & 2.848784923553467                                  \\
  Rumah             & Rumah                        & 0.09352707862854004                           & 2.8643417358398438                                  \\
  Delete            & Delete                        & 0.0890190601348877                            & 2.871215343475342                                  \\
  Standby           & Standby                        & 0.08996152877807617                           & 3.3450722694396973                                  \\
  Translate         & Translate                        & 0.08868694305419922                           & 3.3266687393188477                                  \\
  \hline
\end{longtable}

\begin{longtable}{|c|c|c|c|}
  \caption{Pengujian Kedua Model di Kondisi Cahaya 80 lux}
  \label{tb:prediksiremang2}                                   \\
  \hline
  \rowcolor[HTML]{C0C0C0}
  \textbf{Kosakata} & \textbf{Klasifikasi Model} & \textbf{\emph{Processing Time}} & \textbf{\emph{Complete Time}}\\
  \hline
  Maaf              & Maaf                        & 0.09696221351623535                           & 3.2127857208251953                                  \\
  Tolong            & Tolong                        & 0.09327077865600586                           & 2.8273916244506836                                  \\
  Nama              & Nama                        & 0.09208822250366211                           & 2.7649283409118652                                  \\
  Saya              & Saya                        & 0.08996152877807617                           & 3.50247859954834                                  \\
  Siapa              & Siapa                        & 0.08884954452514648                          & 2.910640239715576                                  \\
  Rumah             & Rumah                       & 0.09037542343139648                           & 2.895984649658203                                  \\
  Delete            & Delete                       & 0.0916590690612793                            & 2.876894474029541                                  \\
  Standby           & Standby                      & 0.08996152877807617                           & 2.8268051147460938                                  \\
  Translate         & Translate                     & 0.09454655647277832                           & 2.887272834777832                                  \\
  \hline
\end{longtable}

\begin{longtable}{|c|c|c|c|}
  \caption{Pengujian Ketiga Model di Kondisi Cahaya 80 lux}
  \label{tb:prediksiremang3}                                   \\
  \hline
  \rowcolor[HTML]{C0C0C0}
  \textbf{Kosakata} & \textbf{Klasifikasi Model} & \textbf{\emph{Processing Time}} & \textbf{\emph{Complete Time}}\\
  \hline
  Maaf              & Maaf                        & 0.08908700942993164                           & 2.8675103187561035                                  \\
  Tolong            & Tolong                        & 0.09565138816833496                           & 2.890141010284424                                  \\
  Nama              & Nama                        & 0.09208822250366211                           & 3.3304882049560547                                  \\
  Saya              & Saya                        & 0.08996152877807617                           & 2.884190082550049                                  \\
  Siapa              & Siapa                        & 0.08884954452514648                           & 2.8408455848693848                                  \\
  Rumah             & \textcolor{red}{Delete}                        & 0.08767223358154297                           & 3.3569884300231934                                  \\
  Delete            & Delete                        & 0.0912930965423584                            & 2.8562521934509277                                  \\
  Standby           & Standby                        & 0.08996152877807617                           & 2.8762435913085938                                  \\
  Translate         & Translate                        & 0.09907031059265137                           &2.8639984130859375                                  \\
  \hline
\end{longtable}

Berdasarkan tiga pengujian yang telah dilakukan, didapatkan bahwa hampir keseluruhan klasifikasi model telah sesuai dengan \emph{class} kosakatanya. Hal ini menunjukkan bahwa peningkatan intensitas cahaya berpengaruh dalam proses klasifikasi yang dilakukan oleh model. Adanya peningkatan intensitas cahaya atau semakin terang kondisi ruangan menghasilkan klasifikasi model yang lebih baik dan tepat sesuai dengan kosakata yang selaras dengan gerakannya. Namun, masih terdapat kesalahan model dalam melakukan klasifikasi. Dapat dilihat pada tabel \ref{tb:prediksiremang3}, untuk isyarat kosakata "Rumah" diklasifikasikan sebagai "Delete". Hal ini disebabkan karena adanya kemiripan antara gerakan untuk kosakata "Rumah" dan "Delete" sehingga apabila pengguna melakukan gerakan isyarat dengan tidak mengutamakan keunikan atau \emph{feature}, dapat menghasilkan klasifikasi yang salah. Berdasarkan tabel \ref{tb:prediksiremang1}, tabel \ref{tb:prediksiremang2}, dan tabel tabel \ref{tb:prediksiremang3} menunjukkan bahwa secara keseluruhan model memiliki akurasi klasifikasi sebesar 96.2\%.

Apabila dilihat berdasarkan waktu pemrosesan, rata - rata waktu yang dibutuhkan model untuk menghasilkan klasifikasi bahasa isyarat (\emph{processing time}) adalah 0.091 detik dan rata - rata waktu yang dibutuhkan dalam menghasilkan klasifikasi bahasa isyarat (\emph{complete time}) adalah 2.982  detik. Dapat dilihat bahwa terdapat peningkatan \emph{processing time} dan \emph{complete time} seiring dengan peningkatan intensitas cahaya ruangan, dimana kemampuan kamera dalam menangkap gerakan bahasa isyarat lebih mudah dan jelas lagi. Hal ini juga berkaitan dengan kemampuan program dalam mengekstrak koordinat yang dibutuhkan untuk menghasilkan klasifikasi, serta lama waktu yang dibutuhkan kamera dalam menangkap tiap \emph{frame} yang nantinya digunakan untuk mendapatkan data koordinat menjadi lebih cepat.

\subsection{Pengujian Model di Kondisi Cahaya 125 lux}
\label{sec:analisiscahaya3}

\begin{longtable}{|c|c|c|c|}
  \caption{Pengujian Pertama Model di Kondisi Cahaya 125 lux}
  \label{tb:prediksiterang1}                                   \\
  \hline
  \rowcolor[HTML]{C0C0C0}
  \textbf{Kosakata} & \textbf{Klasifikasi Model} & \textbf{\emph{Processing Time}} & \textbf{\emph{Complete Time}}\\
  \hline
  Maaf              & Maaf                        & 0.09604573249816895                           & 2.820918560028076                                  \\
  Tolong            & Tolong                        & 0.0939791202545166                           & 1.4211559295654297                                  \\
  Nama              & Nama                        & 0.0941765308380127                           & 2.7805566787719727                                  \\
  Saya              & Saya                        & 0.09122896194458008                           & 2.7644705772399902                                  \\
  Siapa              & Siapa                        & 0.09381675720214844                           & 2.788267135620117                                  \\
  Rumah             & Rumah                        & 0.09244728088378906                           & 3.0803489685058594                                  \\
  Delete            & Delete                        & 0.08883857727050781                           & 2.8483128547668457                                  \\
  Standby           & Standby                        & 0.09215426445007324                           & 1.4267277717590332                                  \\
  Translate         & Translate                        & 0.09544491767883301                           & 2.930331230163574                                  \\
  \hline
\end{longtable}

\begin{longtable}{|c|c|c|c|}
  \caption{Pengujian Kedua Model di Kondisi Cahaya 125 lux}
  \label{tb:prediksiterang2}                                   \\
  \hline
  \rowcolor[HTML]{C0C0C0}
  \textbf{Kosakata} & \textbf{Klasifikasi Model} & \textbf{\emph{Processing Time}} & \textbf{\emph{Complete Time}}\\
  \hline
  Maaf              & Maaf                        & 0.09327888488769531                           & 2.730073928833008                                  \\
  Tolong            & Tolong                        & 0.0879817008972168                           & 1.4511394500732422                                  \\
  Nama              & Nama                        & 0.09166264533996582                           & 2.7438855171203613                                  \\
  Saya              & Saya                        & 0.09122896194458008                           & 2.8403663635253906                                  \\
  Siapa              & Siapa                        & 0.09119915962219238                           & 2.913722991943359                                  \\
  Rumah             & Rumah                        & 0.09163784980773926                           & 2.8235578536987305                                  \\
  Delete            & Delete                        & 0.09363150596618652                           & 2.670907974243164                                  \\
  Standby           & Standby                        & 0.09107518196105957                           & 1.3966870307922363                                  \\
  Translate         & Translate                        & 0.10021495819091797                           & 3.096485137939453                                  \\
  \hline
\end{longtable}

\begin{longtable}{|c|c|c|c|}
  \caption{Pengujian Ketiga Model di Kondisi Cahaya 125 lux}
  \label{tb:prediksiterang3}                                   \\
  \hline
  \rowcolor[HTML]{C0C0C0}
  \textbf{Kosakata} & \textbf{Klasifikasi Model} & \textbf{\emph{Processing Time}} & \textbf{\emph{Complete Time}}\\
  \hline
  Maaf              & Maaf                        & 0.09278368949890137                           & 2.820918560028076                                  \\
  Tolong            & Tolong                        & 0.09278368949890137                           & 1.460716724395752                                  \\
  Nama              & Nama                        & 0.0939791202545166                           & 2.825596332550049                                  \\
  Saya              & Saya                        & 0.09705543518066406                           & 2.8554511070251465                                  \\
  Siapa              & Siapa                        & 0.08928465843200684                           & 2.6734185218811035                                  \\
  Rumah             & Rumah                        & 0.09006929397583008                           & 2.763504981994629                                  \\
  Delete            & Delete                        & 0.09212327003479004                           & 2.730696201324463                                  \\
  Standby           & Standby                        & 0.09167933464050293                           & 1.4569830894470215                                  \\
  Translate         & Translate                        & 0.092132568359375                           & 2.909302711486816                                  \\
  \hline
\end{longtable}

Berdasarkan tiga pengujian yang telah dilakukan, didapat bahwa keseluruhan klasifikasi model telah sesuai dengan \emph{class} kosakatanya. Hal ini menunjukkan bahwa semakin terang atau peningkatan intensitas cahaya berpengaruh dalam proses klasifikasi yang dilakukan oleh model. Pada tingkat intensitas cahaya tertinggi pada pengujian ini, dihasilkan klasifikasi yang baik dan tepat untuk seluruh kosakatanya. Berdasarkan data pada tabel \ref{tb:prediksiterang1}, tabel \ref{tb:prediksiterang2}, tabel \ref{tb:prediksiterang3} menunjukkan bahwa model memiliki akurasi klasifikasi sebesar 100\%.

Apabila dilihat berdasarkan waktu pemrosesan, rata - rata waktu yang dibutuhkan model untuk menghasilkan klasifikasi bahasa isyarat (\emph{processing time}) adalah 0.093 detik dan rata - rata waktu yang dibutuhkan dalam menghasilkan klasifikasi bahasa isyarat (\emph{complete time}) adalah 2.519 detik. Dapat dilihat bahwa terdapat peningkatan \emph{complete time} seiring dengan meningkatnya intensitas cahaya ruangan. Hal ini menunjukkan bahwa kemampuan kamera dalam menangkap gerakan bahasa isyarat lebih mudah dan jelas lagi sehingga dalam memproses tiap \emph{frame} yang nantinya digunakan untuk mendapatkan data koordinat menjadi lebih cepat. Namun, kenaikan intensitas cahaya tidak menyebabkan kenaikan terhadap \emph{processing time}. Apabila dibandingkan dengan pengujian pada intensitas cahaya 80 lux, terdapat penurunan sebesar 0.002 detik. Penurunan ini terbilang sangat kecil dan tidak mempengaruhi pengalaman pengguna dalam menggunakan program penerjemah bahasa isyarat Indonesia (BISINDO) secara keseluruhan. Adanya penurunan \emph{processing time} dapat disebabkan oleh ekstraksi koordinat pada tiap \emph{frame} yang lebih baik lagi, dimana untuk 108 koordinat yang digunakan memiliki suatu nilai dan tidak bernilai 0 (pada \emph{framework} Mediapipe apabila suatu koordinat tidak terdeteksi, maka akan otomatis bernilai 0). Hal ini berdampak pada model yang harus memproses lebih banyak lagi untuk menghasilkan klasifikasi bahasa isyarat.

\section{Pengujian Kondisi Jarak}
\label{sec:analisisjarak}

Pada pengujian kondisi jarak ini dilakukan untuk memahami bagaimana performa model pada jarak kamera dengan pengguna yang berbeda - beda. Adapun jarak yang akan digunakan pada pengujian ini adalah 180 cm, 240 cm, dan 300 cm. Variasi jarak ini dipilih dengan mempertimbangkan bahwa bagian kepala hingga tangan pengguna dapat terlihat secara jelas pada kamera sehingga setiap gerakan bahasa isyarat yang akan dilakukan. Untuk menghasilkan klasifikasi model diperlukan 30 \emph{frame} dan untuk setiap \emph{frame} diekstraksi koordinatnya dengan menggunakan \emph{framework} Mediapipe, apabila bagian tubuh pengguna (terkhususnya tangan dan kepala yang dominan digunakan untuk merepresentasikan suatu bahasa isyarat) tidak terlihat pada kamera akan mempengaruhi data koordinat yang didapatkan dan menyulitkan model untuk melakukan klasifikasi dengan tepat. Adapun posisi pengguna dengan jarak masing - masing dapat dlihat pada tabel \ref{tb:kondisijarak}. 

Model penerjemah bahasa Indonesia (BISINDO) yang akan digunakan pada pengujian ini adalah model pada bagian \ref{sec:analisismodel3} karena merupakan model yang menghasilkan klasifikasi yang terbaik jika dibandingkan dengan model lainnya. Untuk setiap variasi jarak akan dilakukan pengujian sebanyak tiga kali dengan kondisi ruangan yang memiliki intensitas cahaya yang terang (berkisar pada 125 lux). Pada setiap pengujian akan dicari hasil klasifikasi model, waktu yang dibutuhkan model untuk menghasilkan klasifikasi bahasa isyarat berdasarkan data koordinat yang diberikan(\emph{processing time}), dan waktu total yang dibutuhkan dalam menghasilkan klasifikasi bahasa isyarat (\emph{complete time}).  

\begin{longtable}{|c|c|}
  \caption{Variasi Kondisi Jarak}
  \label{tb:kondisijarak}                                   \\
  \hline
  \rowcolor[HTML]{C0C0C0}
  \textbf{Jarak} & \textbf{Gambar Kondisi}  \\
  \hline
  180 cm            &  \includegraphics[scale=0.15]{gambar/bab4-jarak180.png}                \\
  \hline
  240 cm            & \includegraphics[scale=0.15]{gambar/bab4-jarak240.png}                 \\
  \hline
  300 cm            & \includegraphics[scale=0.15]{gambar/bab4-jarak300.png}                 \\
  \hline
\end{longtable}

\subsection{Pengujian Jarak 180 cm}
\label{sec:analisisjarak1}

\begin{longtable}{|c|c|c|c|}
  \caption{Pengujian Pertama Model di Kondisi Jarak 180 cm}
  \label{tb:prediksipendek1}                                   \\
  \hline
  \rowcolor[HTML]{C0C0C0}
  \textbf{Kosakata} & \textbf{Klasifikasi Model} & \textbf{\emph{Processing Time}} & \textbf{\emph{Complete Time}}\\
  \hline
  Maaf              & Maaf                        & 0.09432005882263184                           & 2.6262545585632324                                  \\
  Tolong            & Tolong                        & 0.09578323364257812                           & 2.7582406997680664                                  \\
  Nama              & Nama                        & 0.0936741828918457                           & 2.7642273902893066                                  \\
  Saya              & Saya                        & 0.09214234352111816                           & 1.4595723152160645                                  \\
  Siapa              & Siapa                        & 0.09187936782836914                           & 2.6860642433166504                                  \\
  Rumah             & \textcolor{red}{Delete}                        & 0.09518122673034668                           & 2.7240443229675293                                  \\
  Delete            & Delete                        & 0.09378981590270996                           & 1.4715027809143066                                  \\
  Standby           & Standby                        & 0.09064006805419922                           & 1.4081025123596191                                  \\
  Translate         & Translate                        & 0.09035801887512207                           & 2.901949882507324                                  \\
  \hline
\end{longtable}

\begin{longtable}{|c|c|c|c|}
  \caption{Pengujian Kedua Model di Kondisi Jarak 180 cm}
  \label{tb:prediksipendek2}                                   \\
  \hline
  \rowcolor[HTML]{C0C0C0}
  \textbf{Kosakata} & \textbf{Klasifikasi Model} & \textbf{\emph{Processing Time}} & \textbf{\emph{Complete Time}}\\
  \hline
  Maaf              & \textcolor{red}{Tolong}                        & 0.1025094985961914                           & 2.683281898498535                                  \\
  Tolong            & Tolong                        & 0.10531425476074219                           & 2.6761865615844727                                  \\
  Nama              & Nama                        & 0.10347270965576172                           & 2.837998867034912                                  \\
  Saya              & Saya                        & 0.10077977180480957                           & 1.6065859794616701                                  \\
  Siapa              & Siapa                        & 0.09792852401733398                           & 2.89107084274292                                  \\
  Rumah             & Rumah                        & 0.09961485862731934                           & 2.9763364791870117                                  \\
  Delete            & Delete                        & 0.10204243659973145                           & 1.4318633079528809                                  \\
  Standby           & Standby                        & 0.10916757583618164                           & 1.4253973960876465                                  \\
  Translate         & Translate                        & 0.1098337173461914                           & 3.0064201354980464                                  \\
  \hline
\end{longtable}

\begin{longtable}{|c|c|c|c|}
  \caption{Pengujian Ketiga Model di Kondisi Jarak 180 cm}
  \label{tb:prediksipendek3}                                   \\
  \hline
  \rowcolor[HTML]{C0C0C0}
  \textbf{Kosakata} & \textbf{Klasifikasi Model} & \textbf{\emph{Processing Time}} & \textbf{\emph{Complete Time}}\\
  \hline
  Maaf              & Maaf                        & 0.09703993797302246                           & 2.6079654693603516                                  \\
  Tolong            & Tolong                        & 0.11110496520996094                           & 2.688016891479492                                  \\
  Nama              & Nama                        & 0.11299991607666016                           & 2.854478359222412                                  \\
  Saya              & Saya                        & 0.1057591438293457                           & 1.4061927795410156                                  \\
  Siapa              & Siapa                        & 0.10792779922485352                           & 2.819080352783203                                  \\
  Rumah             & Rumah                        & 0.10235428810119629                           & 2.8994321823120117                                  \\
  Delete            & Delete                        & 0.10874819755554199                           & 1.3875889778137207                                  \\
  Standby           & Standby                        & 0.11034560203552246                           & 1.3965654373168945                                  \\
  Translate         & Translate                        & 0.10168957710266113                           & 3.110654354095459                                  \\
  \hline
\end{longtable}

Berdasarkan tiga pengujian yang dilakukan, didapatkan bahwa hampir keseluruhan klasifikasi model yang sesuai dengan \emph{class} kosakata. Namun, terdapat beberapa kesalahan model dalam melakukan klasifikasi. Dapat dilihat pada tabel \ref{tb:prediksipendek1} untuk isyarat kosakata "Rumah" diklasifikasikan sebagai "Delete" dan pada tabel \ref{tb:prediksipendek2} untuk isyarat kosakata "Maaf" diklasifikasikan sebagai "Tolong". Adanya kemiripan antara kosakata menjadi penyebab utama terjadinya kesalahan ini. Kosakata "Rumah" dan "Delete" memiliki kemiripan pada gerakan isyaratnya, dimana kedua kosakata ini sama - sama menggunakan dua tangan dengan gerakan yang mayoritas terjadi pada bagian badan pengguna. Sedangkan untuk kosakata "Maaf" dan "Tolong" memiliki kemiripan pada gerakan isyaratnya dengan kedua kosakata sama - sama menggunakan tangan kanan dengan gerakan akhir yang mayoritas terjadi pada bagian samping wajah pengguna. Namun kemiripan ini tidak sepenuhnya membuat hasil klasifikasi menjadi lebih condong ke suatu kosakata, melainkan dengan pengguna yang memperagakan bahasa isyarat dengan mengutamakan "ciri khas" dari masing - masing bahasa isyarat dapat menghasilkan hasil klasifikasi model yang baik. Hal ini dapat dilihat pada tabel \ref{tb:prediksipendek2} menghasilkan keseluruhan klasifikasi yang benar untuk seluruh \emph{class} kosakata. Pada pengujian dengan kondisi jarak 180 cm, berdasarkan data pda tabel \ref{tb:prediksipendek1}, tabel tabel \ref{tb:prediksipendek2}, dan tabel tabel \ref{tb:prediksipendek3} menunjukkan bahwa model memiliki akurasi klasifikasi sebesar 92.5\%.

Apabila dilihat berdasarkan waktu pemrosesan, rata - rata waktu yang dibutuhkan model untuk menghasilkan klasifikasi bahasa isyarat (\emph{processing time}) adalah 0.101 detik dan rata - rata waktu yang dibutuhkan dalam menghasilkan klasifikasi bahasa isyarat (\emph{complete time}) adalah 2.352 detik. Berdasakan data ini, dapat diamati bahwa model memerlukan waktu yang terbilang singkat dalam memproses serangkaian data koordinat yang diberikan dengan program penerjemah yang mampu menyelesaikan proses klasifikasi dengan cepat untuk sistem yang berjalan secara \emph{real time}. Meskipun dengan kondisi pengguna yang memiliki jarak yang cukup dekat dengan kamera, tidak mempengaruhi secara signifikan terhadap proses klasifikasi bahasa isyarat yang dilakukan oleh model.

Pada pengujian di kondisi jarak 180 cm ini, perlu diperhatikan bahwa pengguna harus melakukan gerakan bahasa isyarat dengan memastikan bahwa bagian tubuh kepala hingga tangan dengan jelas terlihat. Hal ini sangat krusial karena model memerlukan informasi koordinat yang lengkap untuk setiapp gerakan isyarat sehingga dapat menghasilkan klasifikasi yang tepat. Terkhususnya untuk kosakata "Translate", "Maaf", "Tolong", dan "Delete" yang memiliki gerakan isyarat yang cukup dinamis dan mayoritas gerakannya terjadi di samping kepala pengguna.  

\subsection{Pengujian Jarak 240 cm}
\label{sec:analisisjarak2}

\begin{longtable}{|c|c|c|c|}
  \caption{Pengujian Pertama Model di Kondisi Jarak 240 cm}
  \label{tb:prediksitengah1}                                   \\
  \hline
  \rowcolor[HTML]{C0C0C0}
  \textbf{Kosakata} & \textbf{Klasifikasi Model} & \textbf{\emph{Processing Time}} & \textbf{\emph{Complete Time}}\\
  \hline
  Maaf              & Maaf                        & 0.09363484382629395                           & 2.8636908531188965                                  \\
  Tolong            & Tolong                        & 0.09594297409057617                           & 2.699732780456543                                  \\
  Nama              & Nama                        & 0.0991525650024414                           & 2.716062068939209                                  \\
  Saya              & Saya                        & 0.09490489959716797                           & 1.3848638534545898                                  \\
  Siapa              & Siapa                        & 0.09771037101745605                           & 2.778289318084717                                  \\
  Rumah             & Rumah                        & 0.09178900718688965                           & 2.8422188758850098                                  \\
  Delete            & Delete                        & 0.09782981872558594                           & 2.746889591217041                                  \\
  Standby           & Standby                        & 0.09716463088989258                           & 1.4444875717163086                                  \\
  Translate         & Translate                        & 0.0946202278137207                           & 2.8994321823120117                                  \\
  \hline
\end{longtable}

\begin{longtable}{|c|c|c|c|}
  \caption{Pengujian Kedua Model di Kondisi Jarak 240 cm}
  \label{tb:prediksitengah2}                                   \\
  \hline
  \rowcolor[HTML]{C0C0C0}
  \textbf{Kosakata} & \textbf{Klasifikasi Model} & \textbf{\emph{Processing Time}} & \textbf{\emph{Complete Time}}\\
  \hline
  Maaf              & Maaf                        & 0.10062599182128906                           & 2.837040424346924                                  \\
  Tolong            & Tolong                        & 0.10464692115783691                           & 2.745509147644043                                  \\
  Nama              & Nama                        & 0.0791633129119873                          & 2.7147960662841797                                  \\
  Saya              & Saya                        & 0.11030316352844238                           & 1.4569830894470215                                  \\
  Siapa              & Siapa                        & 0.09940838813781738                           & 2.7176427841186523                                  \\
  Rumah             & Rumah                        & 0.10739994049072266                           & 2.7834320068359375                                  \\
  Delete            & Delete                        & 0.10262727737426758                           & 2.848219871520996                                  \\
  Standby           & Standby                        & 0.10418152809143066                           & 1.4255762100219727                                  \\
  Translate         & Translate                        & 0.10814952850341797                           & 2.913722991943359                                  \\
  \hline
\end{longtable}

\begin{longtable}{|c|c|c|c|}
  \caption{Pengujian Ketiga Model di Kondisi Jarak 240 cm}
  \label{tb:prediksitengah3}                                   \\
  \hline
  \rowcolor[HTML]{C0C0C0}
  \textbf{Kosakata} & \textbf{Klasifikasi Model} & \textbf{\emph{Processing Time}} & \textbf{\emph{Complete Time}}\\
  \hline
  Maaf              & Maaf                        & 0.08161234855651855                           & 2.9344153404235835                                  \\
  Tolong            & Tolong                        & 0.10856509208679199                           & 2.842411994934082                                  \\
  Nama              & Nama                        & 0.08474326133728027                           & 2.8757071495056152                                  \\
  Saya              & Saya                        & 0.10568404197692871                           & 1.7722105979919436                                  \\
  Siapa              & Siapa                        & 0.10201120376586914                           & 2.6211905479431152                                  \\
  Rumah             & \textcolor{red}{Delete}                        & 0.10468339920043945                           & 2.7255606651306152                                  \\
  Delete            & Delete                        & 0.1030888557434082                           & 2.8281641006469727                                  \\
  Standby           & Standby                        & 0.10799098014831543                           & 1.4275431632995605                                  \\
  Translate         & Translate                        & 0.10626840591430664                           & 2.813551425933838                                  \\
  \hline
\end{longtable}

Berdasarkan tiga pengujian yang telah dilakukan, didapatkan bahwa hampir keseluruhan klasifikasi model telah sesuai dengan \emph{class} kosakata yang bersesuaian. Hal ini menunjukkan bahwa adanya pengaruh antara jarak pengguna dengan kamera terhadap proses klasifikasi model. Peningkatan jarak pengguna terhadap menghasilkan klasifikasi yang lebih baik. Namun, masih terdapat kesalahan yang dapat dilihat pada tabel \ref{tb:prediksitengah3}, yaitu isyarat kosakata "Rumah" diklasifikasikan sebagai "Delete". Sama seperti yang sudah dijelaskan pada pengujian di kondisi jarak 180 cm bahwa terdapat kemiripan antara kosakata "Rumah" dengan "Delete". Pada pengujian dengan kondisi jarak 240 cm ini, berdasarkan data pada tabel \ref{tb:prediksitengah1}, tabel \ref{tb:prediksitengah2}, dan tabel \ref{tb:prediksitengah3} menunjukkan bahwa model memiliki akurasi klasifikasi sebesar 96.3\%.

Apabila dilihat berdasarkan waktu pemrosesan, rata - rata waktu yang dibutuhkan model untuk menghasilkan klasifikasi bahasa isyarat (\emph{processing time}) adalah 0.099 detik dan rata - rata waktu yang dibutuhkan dalam menghasilkan klasifikasi bahasa isyarat (\emph{complete time}) adalah 2.506 detik. Apabila dibandingkan dengan pengujian sebelumnya (kondisi jarak 180 cm). Terdapat peningkatan pada nilai \emph{processing time} seiring dengan peningkatan jarak kamera dengan pengguna yaitu sebesar 0.002 detik. Sedangkan untuk nilai \emph{complete time} mengalami penurunan seiring dengan peningkatan jarak kamera dengan pengguna, yaitu sebesar 0.154 detik. Hal ini dapat disebabkan oleh adanya pengaruh antara jarak pengguna dengan kamera terhadap bagaimana beban kerja kamera dalam menangkap tiap \emph{frame}.

\subsection{Pengujian Jarak 300 cm}
\label{sec:analisisjarak3}

\begin{longtable}{|c|c|c|c|}
  \caption{Pengujian Pertama Model di Kondisi Jarak 300 cm}
  \label{tb:prediksijauh1}                                   \\
  \hline
  \rowcolor[HTML]{C0C0C0}
  \textbf{Kosakata} & \textbf{Klasifikasi Model} & \textbf{\emph{Processing Time}} & \textbf{\emph{Complete Time}}\\
  \hline
  Maaf              & Maaf                        & 0.09520149230957031                           & 2.738614082336426                                  \\
  Tolong            & Tolong                        & 0.0926811695098877                           & 2.878732681274414                                  \\
  Nama              & Nama                        & 0.0982358455657959                           & 2.915797233581543                                  \\
  Saya              & Saya                        & 0.0943760871887207                           & 2.8119421005249023                                  \\
  Siapa              & Siapa                        & 0.1026923656463623                           & 2.7649641036987305                                  \\
  Rumah             & Rumah                        & 0.09520149230957031                           & 2.740323543548584                                  \\
  Delete            & Delete                        & 0.09269595146179199                           & 2.981078624725342                                  \\
  Standby           & Standby                        & 0.09149551391601562                           & 2.813115119934082                                  \\
  Translate         & Translate                        & 0.09415459632873535                           & 2.848083972930908                                  \\
  \hline
\end{longtable}

\begin{longtable}{|c|c|c|c|}
  \caption{Pengujian Kedua Model di Kondisi Jarak 300 cm}
  \label{tb:prediksijauh2}                                   \\
  \hline
  \rowcolor[HTML]{C0C0C0}
  \textbf{Kosakata} & \textbf{Klasifikasi Model} & \textbf{\emph{Processing Time}} & \textbf{\emph{Complete Time}}\\
  \hline
  Maaf              & Maaf                        & 0.11374950408935547                           & 2.8633904457092285                                  \\
  Tolong            & Tolong                        & 0.10336184501647949                           & 2.7457737922668457                                  \\
  Nama              & Nama                        & 0.1011803150177002                           & 2.690098285675049                                  \\
  Saya              & Saya                        & 0.11127400398254395                           & 2.720317840576172                                  \\
  Siapa              & Siapa                        & 0.10362887382507324                           & 2.84954309463501                                  \\
  Rumah             & Rumah                        & 0.10374069213867188                           & 2.8711438179016113                                  \\
  Delete            & Delete                        & 0.10743093490600586                           & 2.7329492568969727                                  \\
  Standby           & Standby                        & 0.10061454772949219                           & 2.903716564178467                                  \\
  Translate         & Translate                        & 0.10336613655090332                           & 2.752346992492676                                  \\
  \hline
\end{longtable}

\begin{longtable}{|c|c|c|c|}
  \caption{Pengujian Ketiga Model di Kondisi Jarak 300 cm}
  \label{tb:prediksijauh3}                                   \\
  \hline
  \rowcolor[HTML]{C0C0C0}
  \textbf{Kosakata} & \textbf{Klasifikasi Model} & \textbf{\emph{Processing Time}} & \textbf{\emph{Complete Time}}\\
  \hline
  Maaf              & Maaf                        & 0.09520149230957031                           & 2.7579760551452637                                  \\
  Tolong            & Tolong                        & 0.0926811695098877                           & 2.6619672775268555                                  \\
  Nama              & Nama                        & 0.0982358455657959                           & 2.8861284255981445                                  \\
  Saya              & Saya                        & 0.0943760871887207                           & 2.800147533416748                                  \\
  Siapa              & Siapa                        & 0.1026923656463623                           & 2.763504981994629                                  \\
  Rumah             & Rumah                        & 0.09520149230957031                           & 2.811462879180908                                  \\
  Delete            & Delete                        & 0.09269595146179199                           & 2.6989388465881348                                  \\
  Standby           & Standby                        & 0.09675002098083496                           & 2.7558088302612305                                  \\
  Translate         & Translate                        & 0.10323047637939453                           & 2.67941951751709                                  \\
  \hline
\end{longtable}

Berdasarkan tiga pengujian yang telah dilakukan, didapatkan bahwa keseluruhan klasifikasi model yang sesuai dengan \emph{class} kosakata. Hal ini menunjukkan bahwa peningkatan jarak berpengaruh pada proses klasifikasi yang dilakukan oleh model. Peningkatan jarak antara pengguna dengan kamera, terkhususnya pada jarak terjauh pada pengujian ini, menghasilkan klasifikasi yang lebih baik dan tepat sesuai dengan kosakata yang bersesuaian dengan gerakannya. Hal ini dapat disebabkan oleh semakin jauh jarak antara kamera dengan pengguna memudahkan dalam memproses pose pengguna sehingga menghasilkan data koordinat yang lebih baik lagi. Berdasarkan data pada tabel \ref{tb:prediksijauh1}, tabel \ref{tb:prediksijauh2}, tabel \ref{tb:prediksijauh3} menunjukkan bahwa model memiliki akurasi klasifikasi sebesar 100\%.

Apabila dilihat berdasarkan waktu pemrosesan, rata - rata waktu yang dibutuhkan model untuk menghasilkan klasifikasi bahasa isyarat (\emph{processing time}) adalah 0.099 detik dan rata - rata waktu yang dibutuhkan dalam menghasilkan klasifikasi bahasa isyarat (\emph{complete time}) adalah 2.794 detik. Dapat dilihat bahwa tidak terdapat peningkatan pada \emph{processing time} jika dibandingkan dengan variasi jarak pada pengujian sebelumnya. Namun, pada nilai \emph{complete time} terdapat penurunan jika dibandingkan dengan pengujian sebelumnya, yaitu sebesar 0.288 detik. Hal ini menunjukkan bahwa terdapat penurunan dari nilai \emph{complete time} seiring dengan peningkatan jarak kamera dengan pengguna. Hal ini dapat kembali lagi menguatkan bahwa adanya pengaruh antara jarak pengguna dengan kamera terhadap bagaimana beban kerja kamera dalam menangkap tiap \emph{frame}.

Secara keseluruhan, dapat dilihat bahwa variasi jarak yang berbeda tidak berpengaruh secara signifikan terhadap hasil klasifikasi model. Hal ini menunjukkan bahwa metode normalisasi data yang digunakan telah berhasil menghasilkan model yang invarian terhadap jarak. Dengan catatan bahwa posisi pengguna (terkhususnya bagian kepala dan tangan) dapat dengan jelas terlihat ketika melakukan gerakan isyarat.  

\section{Pengujian Subjek Berbeda}
\label{sec:analisissubjek}

Pada pengujian dengan menggunakan subjek yang berbeda ini dilakukan untuk memahami bagaimana performa model pada pengguna selain darii penulis. Hal ini dilakukan demi melihat apakah model berhasil beradaptasi terhadap data yang bukan merupakan dataset yang digunakan dalam \emph{training} sehingga kedepannya dapat digunakan oleh kalangan luas. Adapun subjek yang akan diujikan berjumlah 2, yaitu 1 perempuan dan 1 laki - laki. Adapun gambaran subjek yang akan diujikan dapat dilihat pada tabel \ref{tb:kondisisubjek}. 

Model penerjemah bahasa Indonesia (BISINDO) yang akan digunakan pada pengujian ini adalah model pada bagian \ref{sec:analisismodel3} karena merupakan model yang menghasilkan klasifikasi yang terbaik jika dibandingkan dengan model lainnya. Untuk setiap intensitas cahaya akan dilakukan pengujian sebanyak tiga kali dengan jarak terhadap kamera sebesar 300 cm dan intensitas cahaya yang berkisar pada nilai 125 lux atau kondisi ruangan terang. Pada setiap pengujian akan dicari hasil klasifikasi model, waktu yang dibutuhkan model untuk menghasilkan klasifikasi bahasa isyarat berdasarkan data koordinat yang diberikan(\emph{processing time}), dan waktu total yang dibutuhkan dalam menghasilkan klasifikasi bahasa isyarat (\emph{complete time}).  

\begin{longtable}{|c|c|}
  \caption{Variasi Subjek Berbeda}
  \label{tb:kondisisubjek}                                   \\
  \hline
  \rowcolor[HTML]{C0C0C0}
  \textbf{Jenis Kelamin} & \textbf{Gambar Subjek}  \\
  \hline
  Perempuan            &  \includegraphics[scale=0.3]{gambar/bab4-rani.png}                \\
  \hline
  Laki - Laki            & \includegraphics[scale=0.3]{gambar/bab4-evan.png}                 \\
  \hline
\end{longtable}

\subsection{Pengujian Subjek Perempuan}
\label{sec:analisisperempuan}

\begin{longtable}{|c|c|c|c|}
  \caption{Pengujian Pertama Model di Subjek Berbeda Perempuan}
  \label{tb:prediksiperempuan1}                                   \\
  \hline
  \rowcolor[HTML]{C0C0C0}
  \textbf{Kosakata} & \textbf{Klasifikasi Model} & \textbf{\emph{Processing Time}} & \textbf{\emph{Complete Time}}\\
  \hline
  Maaf              & \textcolor{red}{Tolong}                        & 0.09316182136535645                           & 3.208651542663574                                  \\
  Tolong            & Tolong                        & 0.0934598445892334                           & 2.874863147735596                                  \\
  Nama              & Nama                        & 0.09218382835388184                           & 3.2549500465393066                                  \\
  Saya              & Saya                        & 0.09529590606689453                           & 3.128707408905029                                  \\
  Siapa              & Siapa                        & 0.09517526626586914                           & 2.9302382469177246                                  \\
  Rumah             & \textcolor{red}{Delete}                        & 0.1019589900970459                           & 3.019108772277832                                  \\
  Delete            & Delete                        & 0.10392284393310547                           & 3.0571460723876953                                  \\
  Standby           & Standby                        & 0.09529590606689453                           & 1.4797425270080564                                  \\
  Translate         & Translate                        & 0.0931706428527832                           & 2.943220138549805                                  \\
  \hline
\end{longtable}

\begin{longtable}{|c|c|c|c|}
  \caption{Pengujian Kedua Model di Subjek Berbeda Perempuan}
  \label{tb:prediksiperempuan2}                                   \\
  \hline
  \rowcolor[HTML]{C0C0C0}
  \textbf{Kosakata} & \textbf{Klasifikasi Model} & \textbf{\emph{Processing Time}} & \textbf{\emph{Complete Time}}\\
  \hline
  Maaf              & Maaf                        & 0.09658217430114746                           & 3.1087589263916016                                  \\
  Tolong            & Tolong                        & 0.0983424186706543                           & 2.8928446769714355                                  \\
  Nama              & Nama                        & 0.09464740753173828                           & 2.980277538299561                                  \\
  Saya              & Saya                        & 0.09416389465332031                           & 2.925252914428711                                  \\
  Siapa              & Siapa                        & 0.0978078842163086                           & 2.927734851837158                                  \\
  Rumah             & Rumah                        & 0.09566903114318848                           & 3.064112663269043                                  \\
  Delete            & Delete                        & 0.10792422294616699                           & 3.3718514442443848                                  \\
  Standby           & Standby                        & 0.09643745422363281                           & 1.5385866165161133                                  \\
  Translate         & Translate                        & 0.10363197326660156                           & 2.9689049720764165                                  \\
  \hline
\end{longtable}

\begin{longtable}{|c|c|c|c|}
  \caption{Pengujian Ketiga Model di Subjek Berbeda Perempuan}
  \label{tb:prediksiperempuan3}                                   \\
  \hline
  \rowcolor[HTML]{C0C0C0}
  \textbf{Kosakata} & \textbf{Klasifikasi Model} & \textbf{\emph{Processing Time}} & \textbf{\emph{Complete Time}}\\
  \hline
  Maaf              & Maaf                        & 0.09667658805847168                           & 3.103616237640381                                  \\
  Tolong            & Tolong                        & 0.09738278388977051                           & 3.0499148368835454                                  \\
  Nama              & Nama                        & 0.09640955924987793                           & 2.927877902984619                                  \\
  Saya              & Saya                        & 0.0947885513305664                           & 1.4657378196716309                                  \\
  Siapa              & Siapa                        & 0.09821367263793945                           & 3.0816650390625                                  \\
  Rumah             & Rumah                        & 0.09819889068603516                           & 2.870771884918213                                  \\
  Delete            & Delete                        & 0.11416792869567871                           & 3.0212616920471187                                  \\
  Standby           & Standby                        & 0.09473133087158203                           & 1.5490007400512695                                  \\
  Translate         & Translate                        & 0.09832525253295898                           & 3.1276202201843266                                  \\
  \hline
\end{longtable}

Berdasarkan tiga pengujian yang telah dilakukan, didapatkan bahwa hampir keseluruhan klasifikasi model yang sesuai dengan \emph{class} kosakata. Namun, terdapat beberapa kesalahan model dalam melakukan klasifikasi. Dapat dilihat pada tabel \ref{tb:prediksiperempuan1} untuk isyarat kosakata "Maaf" diklasifikasikan sebagai "Tolong". Hal ini dapat disebabkan oleh kesalahan pola gerakan isyarat yang digunakan, dimana kurang ditekankannya keunikan atau \emph{feature} dari masing - masing gerakan bahasa isyarat.  Secara garis besar, hasil pengujian ini menunjukkan bahwa model dengan mudah dapat mengklasifikasikan bahasa isyarat yang diperagakan oleh pengguna dengan jenis kelamin perempuan dengan akurasi sebesar 92.5\%. 

Apabila dilihat berdasarkan waktu pemrosesan, rata - rata waktu yang dibutuhkan model untuk menghasilkan klasifikasi bahasa isyarat (\emph{processing time}) adalah 0.098 detik dan rata - rata waktu yang dibutuhkan dalam menghasilkan klasifikasi bahasa isyarat (\emph{complete time}) adalah 2.810 detik. Hal ini menunjukkan bahwa pada subjek perempuan yang berbeda dengan penulis tidak mempengaruhi nilai \emph{processing time} dan \emph{complete time}. Juga data ini semakin menguatkan bahwa model telah dapat beradaptasi dengan subjek yang berbeda dengan penulis.  

\subsection{Pengujian Subjek Laki - Laki}
\label{sec:analisislaki}

\begin{longtable}{|c|c|c|c|}
  \caption{Pengujian Pertama Model di Subjek Berbeda Laki - Laki}
  \label{tb:prediksilaki1}                                   \\
  \hline
  \rowcolor[HTML]{C0C0C0}
  \textbf{Kosakata} & \textbf{Klasifikasi Model} & \textbf{\emph{Processing Time}} & \textbf{\emph{Complete Time}}\\
  \hline
  Maaf              & \textcolor{red}{Tolong}                        & 0.10172176361083984                           & 3.0670595169067383                                  \\
  Tolong            & Tolong                        & 0.09742546081542969                           & 3.017871379852295                                  \\
  Nama              & Nama                        & 0.09616494178771973                           & 3.0398440361022954                                  \\
  Saya              & Saya                        & 0.10310554504394531                           & 1.4680695533752441                                  \\
  Siapa              & Siapa                        & 0.10661101341247559                           & 2.922499179840088                                  \\
  Rumah             & \textcolor{red}{Delete}                        & 0.08926630020141602                           & 2.829294204711914                                  \\
  Delete            & Delete                         & 0.09601879119873047                           & 3.159763813018799                                  \\
  Standby           & Standby                        & 0.09913754463195801                            & 1.5385866165161133                                  \\
  Translate         & Translate                        & 0.10099458694458008                           & 2.9112768173217773                                  \\
  \hline
\end{longtable}

\begin{longtable}{|c|c|c|c|}
  \caption{Pengujian Kedua Model di Subjek Berbeda Laki - Laki}
  \label{tb:prediksilaki2}                                   \\
  \hline
  \rowcolor[HTML]{C0C0C0}
  \textbf{Kosakata} & \textbf{Klasifikasi Model} & \textbf{\emph{Processing Time}} & \textbf{\emph{Complete Time}}\\
  \hline
  Maaf              & Maaf                        & 0.10624909400939941                           & 2.9753994941711426                                  \\
  Tolong            & Tolong                        & 0.09604954719543457                           & 2.9878664016723637                                  \\
  Nama              & Nama                        & 0.09938287734985352                           & 2.855043411254883                                  \\
  Saya              & Saya                        & 0.09951472282409668                           & 1.5473628044128418                                  \\
  Siapa              & Siapa                        & 0.08862924575805664                           & 2.881293296813965                                  \\
  Rumah             & Rumah                        & 0.08926630020141602                           & 2.879476547241211                                  \\
  Delete            & Delete                        & 0.09708309173583984                           & 3.097364902496338                                  \\
  Standby           & Standby                        & 0.10267424583435059                           & 1.5490007400512695                                  \\
  Translate         & Translate                        & 0.10302090644836426                           & 3.1186795234680176                                  \\
  \hline
\end{longtable}

\begin{longtable}{|c|c|c|c|}
  \caption{Pengujian Ketiga Model di Subjek Berbeda Laki - Laki}
  \label{tb:prediksilaki3}                                   \\
  \hline
  \rowcolor[HTML]{C0C0C0}
  \textbf{Kosakata} & \textbf{Klasifikasi Model} & \textbf{\emph{Processing Time}} & \textbf{\emph{Complete Time}}\\
  \hline
  Maaf              & Maaf                        & 0.09524083137512207                           & 3.0119776725769043                                  \\
  Tolong            & Tolong                        & 0.10116004943847656                           & 3.0259251594543457                                  \\
  Nama              & Nama                        & 0.10364699363708496                           & 2.9966139793395996                                  \\
  Saya              & Saya                        & 0.0972299575805664                           & 1.4724183082580566                                  \\
  Siapa              & Siapa                        & 0.09995222091674805                           & 2.967231273651123                                  \\
  Rumah             & Rumah                        & 0.08926630020143602                           & 2.9637622833251953                                  \\
  Delete            & Delete                        & 0.09006118774414062                           & 2.9045462608337402                                  \\
  Standby           & Standby                        & 0.10000467300415039                           & 2.028508186340332                                  \\
  Translate         & Translate                        & 0.10082817077636719                           & 3.1959056854248047                                  \\
  \hline
\end{longtable}

Berdasarkan tiga pengujian yang telah dilakukan, didapatkan bahwa hampir keseluruhan klasifikasi model yang sesuai dengan \emph{class} kosakata. Namun, terdapat beberapa kesalahan model dalam melakukan klasifikasi. Dapat dilihat pada tabel \ref{tb:prediksilaki1} untuk kosakata isyarat "Maaf" diklasifikasikan sebagai "Tolong". Hal ini dapat disebabkan oleh kesalahan pola gerakan isyarat yang digunakan, dimana kurang ditekankannya keunikan atau \emph{feature} dari masing - masing gerakan bahasa isyarat.  Secara garis besar, hasil pengujian ini menunjukkan bahwa model dengan mudah dapat mengklasifikasikan bahasa isyarat yang diperagakan oleh pengguna dengan jenis kelamin perempuan dengan akurasi sebesar 92.5\%. 

Apabila dilihat berdasarkan waktu pemrosesan, rata - rata waktu yang dibutuhkan model untuk menghasilkan klasifikasi bahasa isyarat (\emph{processing time}) adalah 0.098 detik dan rata - rata waktu yang dibutuhkan dalam menghasilkan klasifikasi bahasa isyarat (\emph{complete time}) adalah 2.682 detik. Hal ini menunjukkan bahwa pada subjek yang berbeda dengan penulis tidak mempengaruhi \emph{processing time} dan \emph{complete time}. Data ini juga menunjukkan bahwa proses normalisasi yang dilakukan telah berhasil dan memudahkan model untuk mengklasifikasikan gerakan bahasa isyarat secara \emph{general} atau menyeluruh.  

\section{Pengujian Pembentukan Kalimat dan Konversi Suara}
\label{sec:analisiskalimat}

Pada pengujian pembentukan kalimat dan konversi suara ini dilakukan untuk memahami bagaimana sistem penerjemah bahasa isyarat Indonesia (BISINDO) jika digunakan untuk membentuk kalimat dan melakukan konversi suara berdasarkan kalimat yang dibentuk. Adapun sistematika dalam pembentukan kalimat dan konversi ini mengacu dengan \emph{flowchart} yang telah dijelaskan pada sub bab \ref{sec:metodologisistemkontrol}. Kombinasi kata untuk membentuk kalimat pada sistem ini dapat dilihat pada tabel \ref{tb:kalimatpengujian}. Perlu diperhatikan bahwa untuk mendeteksi kosakata dan mengontrol sistem (\emph{translate} dan \emph{delete}) memerlukan pengguna untuk berada di keadaan \emph{standby} terlebih dahulu. Pembentukan kalimat dan konveri suara pada pengujian ini dilakukan secara sekuensial dengan memastikan bahwa setiap kosakata benar terklasifikasi. Kemudian, kalimat akan diterjemahkan dengan melakukan gerakan isyarat kontrol "\emph{translate}" dan diakhiri dengan melakukan gerakan isyarat kontrol "\emph{delete}" untuk memastikan bahwa kosakata terakhir berhasil dihapus. Pengulangan bahasa isyarat dilakukan maksimal sebanyak tiga kali.

Model penerjemah bahasa Indonesia (BISINDO) yang akan digunakan pada pengujian ini adalah model pada bagian \ref{sec:analisismodel3} karena merupakan model yang menghasilkan klasifikasi yang terbaik jika dibandingkan dengan model lainnya. Untuk setiap kombinasi kalimat akan dilakukan pengujian sebanyak satu kali dengan kondisi ruangan yang memiliki intensitas cahaya yang terang (berkisar pada 125 lux) dan jarak kamera dengan pengguna bernilai 300 cm. Pada setiap kombinasi kalimat akan dicari hasil klasifikasi model, waktu yang dibutuhkan model untuk menghasilkan klasifikasi bahasa isyarat berdasarkan data koordinat yang diberikan(\emph{processing time}), dan waktu total yang dibutuhkan dalam menghasilkan klasifikasi bahasa isyarat (\emph{complete time}).  

\begin{longtable}{|c|c|}
  \caption{Kalimat Pengujian}
  \label{tb:kalimatpengujian}                                   \\
  \hline
  \rowcolor[HTML]{C0C0C0}
  \textbf{Kombinasi Kosakata} & \textbf{Kalimat Akhir}  \\
  \hline
  "Maaf" + "Siapa" + "Nama"            &  "Maaf siapa nama kamu?"               \\

  "Maaf" + "Tolong "+ "Saya"            & "Maaf tolong bantu saya"                 \\
  
  "Maaf" + "Rumah" + "Siapa"            & "Maaf ini rumah siapa?"                 \\

  "Rumah" + "Saya"            & "Ini rumah saya"                 \\

  "Rumah" + "Siapa"            & "Ini rumah siapa"                 \\

  "Siapa" + "Nama"            & "Siapa nama kamu?"                 \\

  "Tolong" + "Saya"            & "Tolong bantu saya"                 \\
  \hline
\end{longtable}

\begin{longtable}{|c|c|c|c|}
  \caption{Pengujian Pembentukan Kalimat Pertama}
  \label{tb:prediksikombinasi1}                                   \\
  \hline
  \rowcolor[HTML]{C0C0C0}
  \textbf{Kosakata} & \textbf{Klasifikasi Model} & \textbf{\emph{Processing Time}} & \textbf{\emph{Complete Time}}\\
  \hline
  Maaf              & Maaf                        & 0.09733414649963379                           & 3.103902339935303                                  \\
  Siapa            & Siapa                        & 0.0981595516204834                           & 2.81724214553833                                  \\
  Nama              & Nama                        & 0.09517359733581543                           & 3.097071647644043                                  \\
  Delete              & Delete                        & 0.09404182434082031                           & 3.1426548957824707                                  \\
  Translate              & Translate                        & 0.09448504447937012                           & 2.94607400894165                                  \\
  \hline
\end{longtable}

\begin{longtable}{|c|c|c|c|}
  \caption{Pengujian Pembentukan Kalimat Kedua}
  \label{tb:prediksikombinasi2}                                   \\
  \hline
  \rowcolor[HTML]{C0C0C0}
  \textbf{Kosakata} & \textbf{Klasifikasi Model} & \textbf{\emph{Processing Time}} & \textbf{\emph{Complete Time}}\\
  \hline
  Maaf              & Maaf                        & 0.10064482688903809                           & 2.822299003601074                                  \\
  Tolong            & Tolong                        & 0.09766435623168945                           & 2.844550609588623                                  \\
  Saya              & Nama                        & 0.09680485725402832                           & 2.9134011268615723                                  \\
  Delete              & Delete                        & 0.09553837776184082                           & 3.1421828269958496                                  \\
  Translate              & Translate                        & 0.09472537040710449                           & 2.8847408294677734                                  \\
  \hline
\end{longtable}

\begin{longtable}{|c|c|c|c|}
  \caption{Pengujian Pembentukan Kalimat Ketiga}
  \label{tb:prediksikombinasi3}                                   \\
  \hline
  \rowcolor[HTML]{C0C0C0}
  \textbf{Kosakata} & \textbf{Klasifikasi Model} & \textbf{\emph{Processing Time}} & \textbf{\emph{Complete Time}}\\
  \hline
  Maaf              & Maaf                        & 0.10550785064697266                           & 2.969491481781006                                  \\
  Rumah            & Tolong                        & 0.0971059799194336                           & 2.9895544052124023                                  \\
  Siapa              & Nama                        & 0.10715579986572266                           & 2.9179072380065922                                  \\
  Delete              & Delete                        & 0.0979464054107666                           & 2.9804134368896484                                  \\
  Translate              & Translate                        & 0.10163474082946777                           & 2.958190441131592                                  \\
  \hline
\end{longtable}

\begin{longtable}{|c|c|c|c|}
  \caption{Pengujian Pembentukan Kalimat Keempat}
  \label{tb:prediksikombinasi4}                                   \\
  \hline
  \rowcolor[HTML]{C0C0C0}
  \textbf{Kosakata} & \textbf{Klasifikasi Model} & \textbf{\emph{Processing Time}} & \textbf{\emph{Complete Time}}\\
  \hline
  Rumah              & \textcolor{red}{Delete}                        & 0.09740185737609863                           & 3.042304515838623                                  \\
  Saya            & Saya                        & 0.0987851619720459                           & 1.4961934089660645                                  \\
  Delete              & Delete                        & 0.09497356414794922                           & 2.9061841964721684                                  \\
  Translate              & Translate                        & 0.10316205024719238                           & 2.943763732910156                                  \\
  \hline
\end{longtable}


\begin{longtable}{|c|c|c|c|}
  \caption{Pengujian Pembentukan Kalimat Kelima}
  \label{tb:prediksikombinasi5}                                   \\
  \hline
  \rowcolor[HTML]{C0C0C0}
  \textbf{Kosakata} & \textbf{Klasifikasi Model} & \textbf{\emph{Processing Time}} & \textbf{\emph{Processing Time}}\\
  \hline
  Rumah              & Rumah                        & 0.09344840049743652                           & 1.5613818168640137                                  \\
  Siapa            & Siapa                        & 0.09488558769226074                           & 1.4586281776428223                                  \\
  Delete              & Delete                        & 0.09873461723327637                           & 3.033885955810547                                  \\
  Translate              & Translate                        & 0.1056978702545166                           & 2.9063844680786133                                  \\
  \hline
\end{longtable}

\begin{longtable}{|c|c|c|c|}
  \caption{Pengujian Pembentukan Kalimat Keenam}
  \label{tb:prediksikombinasi6}                                   \\
  \hline
  \rowcolor[HTML]{C0C0C0}
  \textbf{Kosakata} & \textbf{Klasifikasi Model} & \textbf{\emph{Processing Time}} & \textbf{\emph{Complete Time}}\\
  \hline
  Siapa              & Siapa                        & 0.10315442085266113                           & 3.042304515838623                                  \\
  Nama            & Nama                        & 0.0998544692993164                           & 1.4961934089660645                                  \\
  Delete              & Delete                        & 0.09890484809875488                           & 2.9061841964721684                                  \\
  Translate              & Translate                        & 0.10316205024719238                           & 2.943763732910156                                  \\
  \hline
\end{longtable}

\begin{longtable}{|c|c|c|c|}
  \caption{Pengujian Pembentukan Kalimat Ketujuh}
  \label{tb:prediksikombinasi7}                                   \\
  \hline
  \rowcolor[HTML]{C0C0C0}
  \textbf{Kosakata} & \textbf{Klasifikasi Model} & \textbf{\emph{Processing Time}} & \textbf{\emph{Complete Time}}\\
  \hline
  Tolong              & Tolong                        & 0.10103940963745117                           & 3.0083799362182617                                  \\
  Saya            & Saya                        & 0.10042452812194824                           & 2.944457530975342                                  \\
  Delete              & Delete                        & 0.0952444076538086                           & 3.03605318069458                                  \\
  Translate              & Translate                        & 0.10302400588989258                           & 3.0155038833618164                                  \\
  \hline
\end{longtable}

% \begin{longtable}{|c|c|}
%   \caption{Waktu \emph{Translate} Kombinasi Kalimat}
%   \label{tb:waktutranslate}                                   \\
%   \hline
%   \rowcolor[HTML]{C0C0C0}
%   \textbf{Kalimat} & \textbf{Waktu Penyuaraan Kalimat}\\
%   \hline
%   Maaf siapa nama kamu?              & 1.9165489673614502                               \\
%   Maaf tolong bantu saya            & 1.824713945388794                            \\
%   Maaf ini rumah siapa?              & 1.9595482349395752                         \\
%   Ini rumah saya              & 1.5204029083251953                             \\
%   Ini rumah siapa?              & 1.605020523071289                             \\
%   Siapa nama kamu?              & 1.606485366821289                             \\
%   Tolong bantu saya              & 1.5231249332427979                             \\
%   \hline
% \end{longtable}

Untuk kalimat yang terdiri dari kombinasi tiga kosakata, yaitu kalimat pertama, kalimat kedua, dan kalimat ketiga telah berhasil dibentuk. Hal ini dapat dilihat dari untuk kombinasi kosakata pembentuk kalimat tersebut telah berhasil diklasifikasikan dengan akurasi sebesar 100\% (dapat dilihat pada tabel \ref{tb:prediksikombinasi1}, tabel \ref{tb:prediksikombinasi2}, dan tabel \ref{tb:prediksikombinasi3}). Pada kalimat pertama, rata - rata \emph{processing time} bernilai 0.096 detik dan \emph{complete time} bernilai 3.021 detik. Pada kalimat kedua, rata - rata \emph{processing time} bernilai 0.097 detik dan \emph{complete time} bernilai 2.921 detik. Pada kalimat ketiga, rata - rata \emph{processing time} bernilai 0.102 detik dan \emph{complete time} bernilai 2.963 detik. Dapat dilihat bahwa meskipun gerakan bahasa isyarat dilakukan secara sekuensial, tetap dapat menghasilkan klasifikasi akurasi yang baik.

Untuk kalimat yang terdiri dari kombinasi dua kosakata, yaitu kalimat kelima, keenam, dan ketujuh telah berhasil dibentuk. Hal ini dapat dilihat dari kombinasi kosakata pembentuk kalimat tersebut telah berhasil diklasifikasikan dengan akurasi sebesar 100\% (dapat dilihat pada tabel \ref{tb:prediksikombinasi5}, \ref{tb:prediksikombinasi6}, \ref{tb:prediksikombinasi7}). Pada kalimat kelima, rata - rata \emph{processing time} bernilai 0.098 detik dan \emph{complete time} bernilai 2.240 detik. Pada kalimat keenam, rata - rata \emph{processing time} bernilai 0.100 detik dan \emph{complete time} bernilai 2.199 detik.Pada kalimat ketujuh, rata - rata \emph{processing time} bernilai 0.100 detik dan \emph{complete time} bernilai 3.001 detik. Dapat dilihat bahwa untuk \emph{complete time} relatif lebih cepat pada kalimat dengan kombinasi tiga kosakata jika dibandingkan dengan kalimat dengan kombinasi dua kosakata.

Namun, pada kalimat keempat ("Ini rumah saya") terdapat kosakata yang salah sehingga menyebabkan gagalnya terbentuk kalimat tersebut dalam pengujian yang dilakukan secara sekue\\nsial. Isyarat kosakata "Rumah" diklasifikasikan sebagai "Delete". Hal ini dapat disebabkan oleh adanya kemiripan gerakan antara kedua kosakata tersebut. rata - rata \emph{processing time} bernilai 0.099 detik dan \emph{complete time} bernilai 2.597 detik.  

% Untuk kalimat yang terdiri dari kombinasi tiga kosakata, pada kombinasi kalimat pertama didapat bahwa keseluruhan kosakata bahasa isyarat untuk kalimat "Maaf siapa nama kamu?" telah berhasil dibentuk. Hal ini dapat dilihat untuk seluruh kombinasi kosakata yang telah berhasil diklasifikasikan dengan akurasi sebesar 100\%. Untuk rata - rata \emph{processing time} dan \emph{complete time} adalah 0.096 detik  dan 3.021 detik. Pada kombinasi kalimat kedua, keseluruhan kosakata bahasa isyarat untuk kalimat "Maaf tolong bantu saya?" telah berhasil dibentuk. Hal ini dapat dilihat untuk seluruh kombinasi kosakata yang telah berhasil diklasifikasikan dengan akurasi sebesar 100\%. Untuk rata - rata \emph{processing time} dan \emph{complete time} adalah 0.097 detik  dan 2.921 detik. Pada kombinasi kalimat ketiga, keseluruhan kosakata bahasa isyarat untuk kalimat "Maaf ini rumah siapa?" telah berhasil dibentuk. Hal ini dapat dilihat untuk seluruh kombinasi kosakata yang telah berhasil diklasifikasikan dengan akurasi sebesar 100\%. Untuk rata - rata \emph{processing time} dan \emph{complete time} adalah 0.102 detik  dan 2.963 detik.

% Untuk kalimat yang terdiri dari kombinasi dua kosakata, pada kombinasi kalimat keempat didapat bahwa keseluruhan kosakata bahasa isyarat untuk kalimat "Ini rumah saya" tidak berhasil dibentuk. Hal ini disebabkan oleh kosakata "Rumah" diklasifikasikan sebagai "Delete". Untuk rata - rata \emph{processing time} dan \emph{complete time} adalah 0.096 detik dan 3.021 detik. Pada kombinasi kalimat keenam, keseluruhan bahasa isyarat untuk kalimat "Maaf tolong bantu saya"  telah berhasil dibentuk. Hal ini dapat dilihat untuk seluruh kombinasi kosakata telah berhasil diklasifikasikan dengan akurasi sebesar 100\%.  Untuk rata - rata \emph{processing time} dan \emph{complete time} adalah 0.098 detik  dan 2.240 detik.

% \section{Pengujian Performa NUC}
% \label{sec:analisisnuc}
% \section{Evaluasi Pengujian}
% \label{sec:analisispengujian}

% Dari pengujian yang \lipsum[1]

% % Contoh pembuatan tabel


% \lipsum[2-4]
