\chapter{PENUTUP}
\label{chap:penutup}

% Ubah bagian-bagian berikut dengan isi dari penutup
Pada bab ini akan dipaparkan kesimpulan dari hasil pengujian yang telah dilakukan. Kesimpulan ini akan menjawab permasalahan yang diangkat dengan merujuk pada tujuan dari pelaksanaan tugas akhir ini. Pada bab ini juga dipaparkan saran mengenai hal yang dapat dilakukan untuk mengembangkan penelitian dengan topik yang sama.

\section{Kesimpulan}
\label{sec:kesimpulan}

Berdasarkan hasil pengujian yang telah dilakukan dalam tugas akhir ini, dapat diambil kesimpulan sebagai berikut:

\begin{enumerate}[nolistsep]
  
  \item Sistem penerjemah bahasa isyarat telah berhasil diimplementasikan pada Intel \emph{Next Unit Computing} (NUC) dan dapat berjalan secara \emph{real time}. 
  \item Model LSTM memiliki performa paling baik dengan penggunaan \emph{layer TimeDistributed} yang diikuti dengan 2 \emph{layer} LSTM pada akurasi 99\%.
  \item Intensitas cahaya 125 lux atau kondisi ruangan terang menghasilkan klasifikasi terbaik dengan akurasi sebesar 100\% dan dengan nilai rata - rata FPS yang relatif tinggi, yaitu 12.905.
  \item Jarak kamera terhadap pengguna sebesar 300 cm menghasilkan klasifikasi terbaik dengan akurasi sebesar sebesar 100\%, namun dengan nilai rata - rata FPS yang relatif rendah, yaitu 10.746.
  \item Model berhasil beradaptasi dengan subjek selain penulis dengan akurasi sebesar 92.5\% untuk subjek perempuan dan laki - laki dengan nilai rata - rata FPS yang relatif tinggi, yaitu bernilai 11.702.
  \item Sistem penerjemah dapat membentuk kalimat dan mengkonversi menjadi suara dengan tingkat keberhasilan sebesar 85.7 dengan rata - rata nilai FPS yang relatif tinggi, yaitu 11.678\%.

\end{enumerate}

\section{Saran}
\label{chap:saran}

Berdasarkan hasil yang diperoleh dari tugas akhir ini, saran yang dapat dipertimbangkan untuk pengembangan lebih lanjut adalah sebagai berikut:

\begin{enumerate}[nolistsep]

  \item Menambahkan variasi jarak, intensitas cahaya, dan subjek berbeda untuk menguji bagaim\\ana kemampuan model dalam beradaptasi dengan serangkaian perubahan yang terjadi.
  \item Mempertimbangkan menggunakan metode lain, seperti CNN-LSTM untuk dapat melakukan ekstraksi \emph{feature} dalam bentuk citra sehingga dapat melakukan klasifikasi gerakan bahasa isyarat dengan lebih baik dan akurat lagi.

\end{enumerate}
